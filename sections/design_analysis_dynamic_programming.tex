

\begin{enumerate}[1)]
    \item Solución diseñada
    
Como solución diseñada, se opto por implementar un algoritmo de programación
dinámica que utiliza el método iterativo bottom up, el cual va construyendo la
solución a medida que resuelve los subproblemas que lo componen. En este caso, el
algoritmo recibe como entrada las cadenas $S_1$ y $S_2$ a comparar y devuelve el costo de edición
mínimo, ademas se implemento una matriz auxiliar la cual va almacenando las operaciones
que producen el costo mínimo.

Para calcular la distancia de edición, el algoritmo calcula todos los subproblemas posibles
y almacena los que producen el costo mínimo en una matriz, donde al completarla,
el resultado estará en la ultima posición.

La solución diseñada fue una iteración sobre al algoritmo de programación dinámica iterativo
propuesto en el 
blog \citeauthor{EditDistance} \cite{EditDistance}
, donde se modifico para agregar la operación de transposición y llevar el registro
de operaciones optimas.

\item Pseudocódigo

\begin{algorithm}[H]
    \SetKwFunction{FMain}{ProgramacionDinamica}
    \SetKwProg{Fn}{Function}{:}{} 
    \DontPrintSemicolon
    \footnotesize
    \Fn{\FMain{$S_1$, $S_2$}}{
            $m \gets |S_1|, \, n \gets |S_2|$\;
            Matriz dp[m][n] = 0; Para todas las casillas de la matriz

            \For{i = 1 to m}{
                dp[i][0] = dp[i-1][0] + costo\_eliminacion($S_1$[i-1])\;
            }
            \For{j = 1 to n}{
                dp[0][j] = dp[0][j-1] +costo\_insercion($S_2$[j-1])\;
            }

            \For{i = 1 to m}{
            \For{j = 1 to n}{
                costoIns, costoDel, costoSub = costo\_insercion($S_2$[j-1])\ , costo\_eliminacion($S_1$[i-1])\, costo\_sustitucion($S_1$[i-1], $S_2$[j-1])\;
                dp[i][j] = dp[i-1][j-1] + costoSub\;

                \If{dp[i][j-1] + costoIns < dp[i][j]}{
                    dp[i][j] = dp[i][j-1] + costoIns\;
                }

                \If{dp[i-1][j] + costoDel < dp[i][j]}{
                    dp[i][j] =  dp[i-1][j] + costoDel\;
                }

                \If{i + 1 < m \textbf{and} j + 1 < n \textbf{and} $S_1$[i] = $S_2$[j + 1] \textbf{and} $S_1$[i + 1] = $S_2$[j]}{
                    \If{dp[i-2][j-2] + costo\_transposicion($S_1$[i-2], $S_1$[i-1])\ < dp[i][j]}{
                        dp[i][j] = dp[i-2][j-2] + costo\_transposicion($S_1$[i-2], $S_1$[i-1])\;
                    }
                }
            }
            }

            \Return{dp[m][n]}\;
    }

\end{algorithm}

\item Ejecución del algoritmo

Para ejemplificar el funcionamiento del algoritmo, usaremos las cadenas
abba y baba, ademas, todos los valores de las operaciones serán 1, excepto
por la operación de transposición que tendrá valor 2, los pasos son los siguientes:

\begin{enumerate}[1)]
    \item Como primer paso, se definen los valores de m, n y la matriz dp la cual almacena
    los resultados parciales, esta se inicializa en tamaño [m+1][n+1], para poder manejar
    casos con cadenas vacías.

\item Luego, se llenan la primera columna y fila de la matriz, con las operaciones
de eliminar y insertar para manejar los casos con cadenas vacías.

\item Seguido se entra al bucle principal, donde para $S_1$[0]='a' y $S_2$[0]='b' se calcularan
todas las operaciones llamando a las funciones de costos, para luego tomar como base la operación se sustitución con la cual
se realizaran las comparaciones para determinar cual de todas las operaciones produce el menor
costo, modificando el valor en la matriz dp.

\item Para el ejemplo propuesto, una vez llenada la matriz dp, se retornara la distancia
de edición minima, la cual corresponde a transponer los primeros dos caracteres y su costo es 1.

\end{enumerate}


    \item Complejidad temporal y espacial
    
    Si consideramos m y n como el tamaño de las cadenas $S_1$ y $S_2$, 
    respectivamente, tenemos que
    el algoritmo propuesto posee una complejidad temporal 
    perteneciente a $O\left(n^{2}\right)$, esto
    se debe a que el algoritmo realiza m x n operaciones con costo lineal, lo que corresponde
    a cada combinación de caracteres de las cadenas de entrada, ademas, por su naturaleza
    de programación dinámica, aprovecha los cálculos previos almacenados en la matriz para 
    evitar recalcular algunos costos.
    
    Al ser un algoritmo almacena los valores parciales en una matriz, 
    la complejidad espacial del algoritmo pertenece a $O\left(n^{2}\right)$ 
    ya que tiene que almacenar m+1 x n+1 resultados, esto ocurre en todos los casos
    ya que siempre se llenara la matriz al completo.
    
    Ademas, se podría considerar la memoria que se usa para almacenar las operaciones
    que producen la distancia de edición minima, la cual funciona igual a la matriz dp,
    por lo tanto se necesaria $O\left(n^{2}\right)$ espacio adicional, por lo cual
    la complejidad espacial del algoritmo se mantiene igual.


    \item Transposiciones y costos variables
    
    La inclusion de la operación de transposición no afecta
    a la complejidad, ya que esta depende solamente del tamaño de la entrada,
    por otro lado, los costos variables tampoco afectan a la complejidad, ya que
    al estar almacenados en un vector, el costo de consultar los valores es lineal.

\end{enumerate}

\subsubsection{Descripción de la solución recursiva}

El problema de distancia de edición extendida se puede resolver
mediante la resolución y composición
de subproblemas, ya que al resolver estos subproblemas obtenemos
su optimo local, lo cual al volver a componer la solución general
nos entrega una solución la cual también es optima, En especifico,
si tenemos $S_1$[0:i] y $S_2$[0:j], para resolver el problema
necesitamos conocer como se transforma $S_1$[0:i-1] a $S_2$[0:j-1], asi sucesivamente.

Esta solución se demuestra mediante el principio de optimalidad de Bellman \cite{optimizaciondp}, el
cual describe lo mencionado.

\subsubsection{Relación de recurrencia}

Dada dos cadenas \( S_1 \) y \( S_2 \), 
donde \( S_1 = S_1[1..m] \) y \( S_2 = S_2[1..n] \),
la distancia de edición extendida 
\( \text{DistanciaExtendida}(i, j) \), para (i,j) hasta (m,n), se define  como:

\[  
\text{DistanciaExtendida}(i, j) = 
\begin{cases}
0 & \text{si } i = 0 \text{ y } j = 0, \\
j \cdot \text{costo\_insercion}(S_2[j-1]) & \text{si } i = 0, \\
i \cdot \text{costo\_eliminacion}(S_1[i-1]) & \text{si } j = 0, \\
\min\left\{
\text{DistanciaExtendida}(i-1, j-1) + \text{costo\_replace}(S_1[i-1], S_2[j-1]), \right. \\
\quad \left. \text{DistanciaExtendida}(i, j-1) + \text{costo\_insert}(S_2[j-1]), \right. \\
\quad \left. \text{DistanciaExtendida}(i-1, j) + \text{costo\_delete}(S_1[i-1]), \right. \\
\quad \left. \text{DistanciaExtendida}(i-2, j-2) + \text{costo\_trans}(S_1[i-2], S_1[i-1]) \right\} 
& \text{si } i > 0 \text{ y } j > 0.
\end{cases}
\]

\subsubsection{Identificación de subproblemas}

Los subproblemas los podemos categorizar con las 4 operaciones, entonces:

\begin{enumerate}
    \item Inserción: Para resolver el costo de inserción con (i,j), se necesita
    calcular la distancia entre los primeros i caracteres de $S_1$ y los primeros
    j-1 caracteres de $S_2$ para luego sumarle el costo de insertar el carácter
    $S_2$[j-1].
    \item Eliminación: Para resolver el costo de eliminación con (i,j), se necesita
    calcular la distancia entre los primeros i-1 caracteres de $S_1$ y los primeros
    j caracteres de $S_2$ para luego sumarle el costo de eliminar el carácter
    $S_1$[i-1].
    \item Sustitución: Para resolver el costo de sustitución con (i,j), se necesita
    calcular la distancia entre los primeros i-1 caracteres de $S_1$ y los primeros
    j-1 caracteres de $S_2$ para luego sumarle el costo de sustituir el carácter
    $S_1$[i-1] por el carácter $S_2$[j-1].
    \item Transposición: Para resolver el costo de transposición con (i,j), se necesita
    calcular la distancia entre los primeros i-2 caracteres de $S_1$ y los primeros
    j-2 caracteres de $S_2$ para luego sumarle el costo de transponer los caracteres
    $S_1$[i-2] y $S_1$[i-1].

\end{enumerate}

Con esto, podemos ver que los subproblemas se solapan, lo que nos da la oportunidad
de optimizar el algoritmo evitando recalcular ciertos resultados.

\subsubsection{Estructura de datos y orden de cálculo}

Como estructura de datos se utilizo una matriz para almacenar los resultados
parciales óptimos, los cuales compondrán el resultado final, ademas esta matriz 
es llenada desde izquierda a derecha y arriba hacia abajo, esto, ya que como definimos
en los subproblemas, se necesitan los resultados anteriores o con menor indice para componer
la solución general.