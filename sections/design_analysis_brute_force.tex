
\begin{enumerate}[1)]
    \item Solución diseñada:

    Se diseñó un algoritmo recursivo de fuerza bruta 
    para calcular la distancia mínima de edición entre dos cadenas, S1 y S2. 
    Este algoritmo utiliza índices i y j
    para comparar caracteres de las cadenas y 
    una lista para registrar las operaciones óptimas que minimizan el 
    costo de transformación.
    
    El algoritmo explora todas las posibles 
    combinaciones de operaciones (inserción, eliminación, sustitución, y transposición) y
    selecciona la que tiene el menor costo, basado en una matriz de costos. 
    La solución diseñada fue una iteración sobre al algoritmo de fuerza bruta
    recursiva propuesto en el 
    blog \citeauthor{EditDistance} \cite{EditDistance}
    , donde se modifico para agregar la operación de transposición y llevar el registro
    de operaciones optimas.


\item Pseudocódigo:


\begin{algorithm}[H]
    \SetKwFunction{FMain}{FuerzaBruta}
    \SetKwProg{Fn}{Function}{:}{}  % Cambia 'AlgorithmName' por el nombre del enfoque elegido

    \DontPrintSemicolon
    \footnotesize

    % Definición del algoritmo principal
    \Fn{\FMain{s1, s2, i, j}}{
        $m \gets |s1|, \, n \gets |s2|$\;
        \If{i = m \textbf{and} j = n}{
            \Return 0\;
        }
    
        \If{i = m}{
            \Return CostoInsertar(s2[j]) + \FMain{s1, s2, i, j + 1}\;
        }
    
        \If{j = n}{
            \Return  CostoEliminar(s1[i]) + \FMain{s1, s2, i + 1, j}\;
        }
    
       
        \If{i < m \textbf{and} j < n \textbf{and} s1[i] = s2[j]}{
            \Return \FMain{s1, s2, i + 1, j + 1}\;
        }
        \Else{
            ins $\leftarrow$ CostoInsertar(s2[j]) + \FMain{s1, s2, i, j + 1}\;
            del $\leftarrow$ CostoEliminar(s1[i]) + \FMain{s1, s2, i + 1, j}\;
            sub $\leftarrow$ CostoSustituir(s1[i], s2[j]) + \FMain{s1, s2, i + 1, j + 1}\;
            trans $\leftarrow$ INT\_MAX\;
    
            \If{i + 1 < m \textbf{and} j + 1 < n \textbf{and} s1[i] = s2[j + 1] \textbf{and} s1[i + 1] = s2[j]}{
                trans $\leftarrow$ CostoTransponer(s1[i], s1[i + 1]) + \FMain{s1, s2, i + 2, j + 2}\;
            }
        }
    
        costo\_min $\leftarrow$ min(\{ins, del, sub, trans\})\;
    
        \Return costo\_min\;
    }


\end{algorithm}

\item Ejecución del algoritmo

Para ejemplificar el funcionamiento del algoritmo, usaremos las cadenas
abba y baba, además, todos los valores de las operaciones serán 1, excepto
por la operación de transposición que tendrá valor 2, los pasos son los siguientes:


\begin{enumerate}[1)]
    \item Como primer paso, se definen los valores de m y n, además se comprueba si alguno de los 
indices ha llegado al final, como no es el caso se sigue la ejecución.

\item Luego, se comparan S1[0]='a' y S2[0]='b', al ser distintos estos ingresan al bloque ELSE 
donde se calcularan el costo de todas las operaciones, este calculo se compone de llamar la función de costo 
respectiva y realizar la llamada recursiva para continuar con el resto de caracteres, 
probando las 4 operaciones por cada indice.

\item Al volver la llamada recursiva a la llamada raíz, se calcula el mínimo
entre las operaciones iniciales, se guarda el valor y se procede a agregar las
acciones a la lista. esto ocurre en cada llamada recursiva.

\item Como resultado de la ejecución, el programa determina que la distancia mínima de edición
que transforma la cadena 'abba' en 'baba' es de 1, lo cual corresponde a realizar una
transición del primer y segundo carácter. En el caso que el valor de transponer fuera 3, el coste mínimo de edición
pasaría a ser producido por la combinación de inserciones y eliminaciones.


\end{enumerate}

\item Complejidad temporal y espacial

Si consideramos m y n como el tamaño de las cadenas S1 y S2, 
respectivamente, tenemos que
el algoritmo propuesto posee una complejidad temporal 
perteneciente a $O\left(4^{\min(n, m)}\right)$, esto
se debe a que el algoritmo tiene que calcular todas las posibles
combinaciones, donde por cada carácter de alguna de las cadenas, existen
4 opciones factibles, además, se usa el mínimo entre los dos tamaños ya que
al llegar al final de una de las cadenas, se sigue completando los caracteres
restantes con operaciones de inserción o eliminación, 
lo cual nos lleva a tiempo lineal.

Al ser un algoritmo que utiliza recursion y además solo almacena variables
de manera constante, la complejidad espacial del algoritmo pertenece a $O\left({\min(n, m)}\right)$ 
donde este mínimo corresponde a la altura del árbol de recursion.

Además, se podría considerar la memoria que se usa para almacenar las operaciones
que producen la distancia de edición mínima, donde el máximo de operaciones
por cada par de cadenas se define como m+n, por lo tanto la memoria necesaria
para almacenar las operaciones en el peor caso pertenece a $O\left({m+n}\right)$


\item Transposiciones y costos variables

En el algoritmo implementado, la complejidad depende directamente
de la cantidad de operaciones disponibles, al agregar la operación 
de transposición, la complejidad aumenta desde $O\left(3^{\min(n, m)}\right)$ a 
$O\left(4^{\min(n, m)}\right)$, por otra parte, los costos variables
modifican las operaciones que producen la distancia de edición minima, pero no aumentan
el orden de la complejidad, ya que estos costos están almacenados en memoria
y recuperar su valor tiene un costo de $O\left({1}\right)$.

\end{enumerate}