
\begin{comment}
\begin{mdframed}
    \textbf{La extensión máxima para esta sección es de 2 páginas.}
\end{mdframed}

La introducción de este tipo de informes o reportes, tiene como objetivo principal \textbf{contextualizar el problema que se va a analizar}, proporcionando al lector la información necesaria para entender la relevancia del mismo. 

Es fundamental que en esta sección se presenten los antecedentes del problema, destacando investigaciones previas o principios teóricos que sirvan como base para los análisis posteriores. Además, deben explicarse los objetivos del informe, que pueden incluir la evaluación de un algoritmo, la comparación de métodos o la validación de resultados experimentales.

Aunque la estructura y el enfoque siguen principios de trabajos académicos, se debe recordar que estos informes no son publicaciones científicas formales, sino trabajos de pregrado. Por lo tanto, se busca un enfoque claro y directo, que permita al lector comprender la naturaleza del problema y los objetivos del análisis, sin entrar en detalles excesivos. 


Introduction Checklist de \citetitle{GoodScientificPaper} \cite{GoodScientificPaper}, adaptada a nuestro contexto:

\begin{itemize}
\item Indique el \textbf{campo del trabajo} (Análisis y Diseño de algoritmos en Ciencias de la Computación), por qué este campo es importante y qué se ha hecho ya en este área, con las \textbf{citas} adecuadas de la literatura académica o fuentes relevantes.
\item Identifique una \textbf{brecha} en el conocimiento, un desafío práctico, o plantee una \textbf{pregunta} relacionada con la eficiencia, complejidad o aplicabilidad de un algoritmo particular.
\item Resuma el propósito del informe e introduzca el análisis o experimento, dejando claro qué se está investigando o comparando, e indique \textbf{qué es novedoso} o por qué es significativo en el contexto de un curso de pregrado.
\item Evite; repetir el resumen; proporcionar información innecesaria o fuera del alcance de la materia (limítese al análisis de algoritmos o conceptos de complejidad); exagerar la importancia del trabajo (recuerde que se trata de un informe de pregrado); afirmar novedad sin una comparación adecuada con lo enseñado en clase o la bibliografía recomendada.
\end{itemize}



\begin{mdframed}
Recuerde que este es su trabajo, y sólo usted puede expresar con precisión lo que ha aprendido y quiere transmitir. Si lo hace bien, su introducción será más significativa y valiosa que cualquier texto automatizado. ¡Confíe en sus habilidades, y verá que puede hacer un mejor trabajo que cualquier herramienta que automatiza la generación de texto!
\end{mdframed}
\end{comment}

\begin{quote}
"La algoritmia es una de las
bases fundamentales de ciencias de la computación. Aunque otras areas
han ganado terreno últimamente e.g., ciencia de datos, inteligencia
artificial y deep learning , la algoritmia sigue siendo fundamental para
proveer soluciones eficientes a muchos de los problemas que aparecen
en esas areas. Es, de alguna manera, un area transversal a ciencias de
la computación."
\end{quote}\citetitle{algoritmos_discretos} \cite{algoritmos_discretos}


En este informe se estudiará la distancia de edición 
extendida o también conocida como
\textbf{Damerau-Levenshtein Distance}, este algoritmo nos permite
calcular el numero mínimo de operaciones para transformar una cadena
de caracteres en otra. Las operaciones corresponden a \textbf{Sustitución,
Inserción, Eliminación y Transposición.} Donde cada una tiene un costo
asociado y el cual se busca minimizar.

El algoritmo tiene diversas aplicaciones, dentro de ellas esta
la búsqueda sobre documentos mediante escaneo, búsqueda en textos antiguos,
búsqueda en bases de datos biológicas y corrección ortográfica. Lo cual
hace que este algoritmo sea igual de importante para la ciencia que para
la vida cotidiana. Ademas, es importante mencionar que debido a la cantidad
de datos a analizar en los distintos campos, la eficiencia del algoritmo
es algo que se ha ganado importancia a traves del tiempo.
\cite{algoritmos_discretos}

Con este informe se busca implementar el algoritmo mediante dos metodologías
de diseño distintas, fuerza bruta y programación dinámica. Las cuales en la
teoría, poseen distintas complejidades temporales y espaciales. Esto se hará
con el fin de comparar empíricamente las dos implementaciones y 
contrastarlo con la teoría, para asi determinar cual de las dos es mejor.

Teóricamente, la implementación del algoritmo mediante fuerza bruta 
posee una complejidad temporal perteneciente a \( O(4^n) \), donde el 4 proviene
de la cantidad de operaciones a probar y el n corresponde al tamaño de la cadena mas larga, la complejidad
espacial del algoritmo pertenece a \( O(m) \) (Que corresponde a la profundidad de la pila
de recursion). Por otra parte, la implementación mediante programación dinámica
posee una complejidad temporal y espacial perteneciente a \( O(n*m) \) donde n y m corresponden
al largo de las cadenas.

Sobre el papel, a medida que crece el tamaño de la entrada, el algoritmo bajo programación
dinámica debería ser mucho mejor, por otra parte, el algoritmo de fuerza bruta debería
utilizar menor cantidad de espacio adicional, por lo cual dependiendo del escenario, uno
podría ser mejor que el otro. Es significativo 
realizar estas comparaciones ya que puede ocurrir 
que en asintóticamente una implementación sea mejor que otra
pero en la practica, por ejemplo, solo se cumpla para entradas muy grandes.
Por lo cual, nos podemos preguntar ¿Esto se refleja en escenarios reales?


Para realizar las comparaciones
se medirá el tiempo de ejecución de cada implementación
al igual que el consumo de memoria RAM para las mismas entradas, con 
el fin de reducir desvíos en las mediciones, se usara el promedio
de varias ejecuciones para dar un resultado mas significativo.
