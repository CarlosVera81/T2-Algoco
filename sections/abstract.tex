
En este reporte se explora la distancia de edición extendida, 
un algoritmo que 
calcula el número mínimo de operaciones necesarias 
para transformar una cadena en otra mediante sustituciones, 
inserciones, eliminaciones y transposiciones. 
El estudio se centra en comparar dos implementaciones:
fuerza bruta y programación dinámica, considerando sus 
complejidades temporales y espaciales tanto en teoría como
en práctica, ademas se busca comparar el uso de costos variables
en las operaciones.

Se diseñaron distintos datasets, 
midiendo el tiempo de ejecución y consumo de memoria
de las implementaciones. Los resultados confirman que la 
programación dinámica supera a la
fuerza bruta en eficiencia para entradas
grandes, mientras que esta última es mejor para entradas
pequeñas. Además, se evidencia que los costos 
variables afectan la elección de las operaciones
óptimas, pero no a la complejidad temporal o espacial.