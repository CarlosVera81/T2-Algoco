\begin{mdframed}
    \textbf{La extensión máxima para esta sección es de 2 páginas.}
\end{mdframed}

Se diseñaron 10 Datasets con casos de prueba, cada dataset contiene 20
pares de cadenas y su distancia de edición esperada, la cual fue calculada
con la librería de python weighted-levenshtein, que permite trabajar con 
el calculo de OSA y costos personalizados

\begin{mdframed}
    Cuando se habla de matrices de costo estándar, esto hace referencia a que 
    los costos de las operaciones son los siguientes:
    \begin{itemize}
        \item Inserción: 1, Para cualquier carácter
        \item Eliminación: 1, Para cualquier carácter
        \item Sustitución: 2, Para cualquier par de caracteres distintos
        \item Transposición: 1, Para cualquier par de caracteres distintos 
    \end{itemize}
\end{mdframed}    

\begin{enumerate}
    \item Datasets
        \begin{itemize}
            \item Transposiciones con matrices de costo estándar: este dataset cuenta con cadenas de largo 8
            , donde son necesarias las transposiciones, con matrices
            de costo estándar.
            \item Transposiciones con matrices de costo con valores modificados para la transposición: este
            dataset cuenta con cadenas de largo 8, donde son necesarias las transposiciones,
            con matrices de costo con valores mas elevados para la operación de transposición.
            \item Cadenas vacías: dataset con una cadena vacía y otra de largo entre 5 y 8, con
            matrices de costo estándar.
            \item Caracteres repetidos: dataset con par de cadenas con caracteres repetidos
            hasta 4 veces, de largo 8, con matrices de costo estándar.
            \item Palabras aleatorias con matrices de costo estándar: diversos datasets con cadenas de tamaño
                menor a 2, 3 a 4, 7, 8 y 12, con matrices de costo estándar
            \item Palabras aleatorias con matrices de costo modificado: dataset con
            cadenas de largo 8, donde las operaciones tienen los costos: Inserción=2,
            Eliminación=2, Sustitución=3, Transposición=2.
        \end{itemize}

\end{enumerate}

Los resultados a los casos de prueba se encuentran en el siguiente recurso, divido
por carpetas donde se almacena la salida del programa bf.cpp, dp.cpp y el dataset en cuestión.

En los archivos de salida, se encuentra las cadenas comparadas, la distancia calculada,
la distancia esperada correcta,una flag que indica si el resultado esta correcto, el tiempo
de ejecución y las operaciones que producen la distancia minima de edición, en conjunto
con sus costos. Ademas, al final del archivo se muestra el tiempo promedio de ejecución del dataset.
