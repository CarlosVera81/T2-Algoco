

El análisis e implementación del algoritmo de distancia de 
edición extendida, mediante las técnicas de fuerza bruta y
programación dinámica, permitieron validar empíricamente lo descrito
por la teoría en base a sus complejidades temporales y espaciales. 
Los resultados demuestran que, mientras el algoritmo basado en 
programación dinámica sobresale en términos de eficiencia temporal,
el uso de memoria aumenta a medida que crece la entrada. Por otra lado,
el algoritmo basado en fuerza bruta 
presenta un desempeño adecuado solo en entradas de tamaño reducido, a 
medida que crece la entrada, la eficiencia de este algoritmo empeora,
pero su uso de memoria se comporta de manera lineal.

Este reporte no solo confirma que la programación dinámica 
es más adecuada para entradas más grandes, sino que también
evidencia cómo la elección del algoritmo depende del contexto, 
ya que el bajo uso de memoria del algoritmo de fuerza bruta podría
ser un factor clave.
Por otro lado, los costos variables sobre la secuencia
de operaciones óptimas no represento un cambio en los tiempos
de ejecución o uso de memoria, reafirmando que estos parámetros 
no afectan a las complejidades, pero sí al comportamiento del algoritmo.  

En resumen, el trabajo logra comparar exitosamente 
las distintas técnicas de diseño de algoritmos, 
comprobando empíricamente la teoría, asi, se puede 
seguir desarrollando en base a lo propuesto para mejorar
estas técnicas.














\begin{comment}

\begin{mdframed}
    \textbf{La extensión máxima para esta sección es de 1 página.}
\end{mdframed}

La conclusión de su informe debe enfocarse en el resultado más importante de su trabajo. No se trata de repetir los puntos ya mencionados en el cuerpo del informe, sino de interpretar sus hallazgos desde un nivel más abstracto. En lugar de describir nuevamente lo que hizo, muestre cómo sus resultados responden a la necesidad planteada en la introducción.

\begin{itemize}
    \item  No vuelva a describir lo que ya explicó en el desarrollo del informe. En cambio, interprete sus resultados a un nivel superior, mostrando su relevancia y significado.
    \item Aunque no debe repetir la introducción, la conclusión debe mostrar hasta qué punto logró abordar el problema o necesidad planteada en el inicio. Reflexione sobre el éxito de su análisis o experimento en relación con los objetivos propuestos.
    \item No es necesario restablecer todo lo que hizo (ya lo ha explicado en las secciones anteriores). En su lugar, centre la conclusión en lo que significan sus resultados y cómo contribuyen al entendimiento del problema o tema abordado.
    \item No deben centrarse en sí mismos o en lo que hicieron durante el trabajo (por ejemplo, evitando frases como "primero hicimos esto, luego esto otro...").
    \item Lo más importante es que no se incluyan conclusiones que no se deriven directamente de los resultados obtenidos. Cada afirmación en la conclusión debe estar respaldada por el análisis o los datos presentados. Se debe evitar extraer conclusiones generales o excesivamente amplias que no puedan justificarse con los experimentos realizados.
\end{itemize}

\end{comment}